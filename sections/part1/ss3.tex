\documentclass[main]{subfiles}

%1.3[Dimensionality Restrictions]

\begin{document}
この章では,最終的に導きたい結論であった4次元においてはRiemann計量による場の方程式4次元ではEinstein方程式が唯一の方程式になることを示していく.実は,これまでの議論は触れてこなかったが次元\(n\)に関する指定を何もしていなかった.つまり,これより前の結果は一般の次元\(n\)で成り立つ結論となっている.

これ以後の議論では,具体的に次元を2次元,3次元,4次元と指定して進めて場の方程式が得ていく.
\subsection{Lemma 4.について}
まずは,簡単な2次元と3次元について検証していく.

\textbf{Lemma 4.}
\(n=2,3\)のいづれの場合で,なおかつ\eqref{lemma1}が満たされるとき,
\begin{equation*}
    \lam{ij,kh;rs,tu}=0
\end{equation*}
となる.これは,\(\lam{ij,kh;rs,tu}\)のもつ対称性\eqref{id1},\eqref{sym1}と\eqref{lemma1}を適応すると\(n=2,3\)の場合にはその多くは重複し結果的に0となることを示せる\textcolor{red}{はずである.鋭意検証中です.}
このことから言える結論は,
\begin{equation*}
    \chi ^{hk,mi;rs,tu}=0
\end{equation*}
ということになる.当然その発散はなくなり,反対称にしようがゼロはゼロななのでLemma 1.と2.は自動的に満たすことなる.つまり,\(n=2,3\)の場合においては花から3次のEL方程式はないという結論となる.しかしこれで満足してはいけない,これよりも強い結論を引き出すことができる.
\subsection{Theorem 2.とその証明}
\textbf{Theorem 2.}\(n=2,3\)のときに限り,2次のEL方程式が得られるLagrangianは以下のみである.
\begin{equation*}
    \mathscr{L} = a\sqrt{g}R + b\sqrt{g}
    \tag{3.1}\label{seclag}
\end{equation*}
ただし,\(a,b=const.\)であり\(R\)はスカラー曲率である.

\textbf{proof.}
Lemma 4.と\eqref{lam2}から,
\begin{equation*}
    \lam{ij,kh;rs,tu}= \dfrac{\partial \lam{ij,kh}}{\partial g_{rs,tu} } =0
\end{equation*}
より,
\begin{equation*}
    \lam{ij,hk} = \lam{ij,hk}(g_{rs},g_{rs,t})
    \tag{3.2}\label{3.2}
\end{equation*}
となる.
ここで,DU PLESSISの([3],p.53)\footnote{この論文が見つからない.}の論文によれば,\(g_{rs,tu}\)に依存しないそれ以下の微分項による任意のテンソルは1次の微分項からも独立しているため,
\begin{equation*}
    \lam{ij,hk} = \lam{ij,hk}(g_{rs})
\end{equation*}
とかける.\textcolor{red}{証明は捜索検討中.}

このように得られた\(\lam{}\)は確定しないように思えるが,\((i,j)\),\((hk)\)のそれぞれの入れ替えに関して対称であり,この組同士の入れ替えも対称,加えて\((jkh)\)のように3つのインデックスに関してもcyclicな対称性を持っている.ここまで強い条件を与え,次元を2,3次と縛ると自ずとその具体的な形が得られることが示せる.

\subsection{Theorem 3.とその証明}

\subsection{Theorem 4.とその証明}

\subsection{Theorem 5.とその証明}

\end{document}
